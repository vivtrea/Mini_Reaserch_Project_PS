% Options for packages loaded elsewhere
\PassOptionsToPackage{unicode}{hyperref}
\PassOptionsToPackage{hyphens}{url}
\documentclass[
]{article}
\usepackage{xcolor}
\usepackage[margin=1in]{geometry}
\usepackage{amsmath,amssymb}
\setcounter{secnumdepth}{-\maxdimen} % remove section numbering
\usepackage{iftex}
\ifPDFTeX
  \usepackage[T1]{fontenc}
  \usepackage[utf8]{inputenc}
  \usepackage{textcomp} % provide euro and other symbols
\else % if luatex or xetex
  \usepackage{unicode-math} % this also loads fontspec
  \defaultfontfeatures{Scale=MatchLowercase}
  \defaultfontfeatures[\rmfamily]{Ligatures=TeX,Scale=1}
\fi
\usepackage{lmodern}
\ifPDFTeX\else
  % xetex/luatex font selection
\fi
% Use upquote if available, for straight quotes in verbatim environments
\IfFileExists{upquote.sty}{\usepackage{upquote}}{}
\IfFileExists{microtype.sty}{% use microtype if available
  \usepackage[]{microtype}
  \UseMicrotypeSet[protrusion]{basicmath} % disable protrusion for tt fonts
}{}
\makeatletter
\@ifundefined{KOMAClassName}{% if non-KOMA class
  \IfFileExists{parskip.sty}{%
    \usepackage{parskip}
  }{% else
    \setlength{\parindent}{0pt}
    \setlength{\parskip}{6pt plus 2pt minus 1pt}}
}{% if KOMA class
  \KOMAoptions{parskip=half}}
\makeatother
\usepackage{color}
\usepackage{fancyvrb}
\newcommand{\VerbBar}{|}
\newcommand{\VERB}{\Verb[commandchars=\\\{\}]}
\DefineVerbatimEnvironment{Highlighting}{Verbatim}{commandchars=\\\{\}}
% Add ',fontsize=\small' for more characters per line
\usepackage{framed}
\definecolor{shadecolor}{RGB}{248,248,248}
\newenvironment{Shaded}{\begin{snugshade}}{\end{snugshade}}
\newcommand{\AlertTok}[1]{\textcolor[rgb]{0.94,0.16,0.16}{#1}}
\newcommand{\AnnotationTok}[1]{\textcolor[rgb]{0.56,0.35,0.01}{\textbf{\textit{#1}}}}
\newcommand{\AttributeTok}[1]{\textcolor[rgb]{0.13,0.29,0.53}{#1}}
\newcommand{\BaseNTok}[1]{\textcolor[rgb]{0.00,0.00,0.81}{#1}}
\newcommand{\BuiltInTok}[1]{#1}
\newcommand{\CharTok}[1]{\textcolor[rgb]{0.31,0.60,0.02}{#1}}
\newcommand{\CommentTok}[1]{\textcolor[rgb]{0.56,0.35,0.01}{\textit{#1}}}
\newcommand{\CommentVarTok}[1]{\textcolor[rgb]{0.56,0.35,0.01}{\textbf{\textit{#1}}}}
\newcommand{\ConstantTok}[1]{\textcolor[rgb]{0.56,0.35,0.01}{#1}}
\newcommand{\ControlFlowTok}[1]{\textcolor[rgb]{0.13,0.29,0.53}{\textbf{#1}}}
\newcommand{\DataTypeTok}[1]{\textcolor[rgb]{0.13,0.29,0.53}{#1}}
\newcommand{\DecValTok}[1]{\textcolor[rgb]{0.00,0.00,0.81}{#1}}
\newcommand{\DocumentationTok}[1]{\textcolor[rgb]{0.56,0.35,0.01}{\textbf{\textit{#1}}}}
\newcommand{\ErrorTok}[1]{\textcolor[rgb]{0.64,0.00,0.00}{\textbf{#1}}}
\newcommand{\ExtensionTok}[1]{#1}
\newcommand{\FloatTok}[1]{\textcolor[rgb]{0.00,0.00,0.81}{#1}}
\newcommand{\FunctionTok}[1]{\textcolor[rgb]{0.13,0.29,0.53}{\textbf{#1}}}
\newcommand{\ImportTok}[1]{#1}
\newcommand{\InformationTok}[1]{\textcolor[rgb]{0.56,0.35,0.01}{\textbf{\textit{#1}}}}
\newcommand{\KeywordTok}[1]{\textcolor[rgb]{0.13,0.29,0.53}{\textbf{#1}}}
\newcommand{\NormalTok}[1]{#1}
\newcommand{\OperatorTok}[1]{\textcolor[rgb]{0.81,0.36,0.00}{\textbf{#1}}}
\newcommand{\OtherTok}[1]{\textcolor[rgb]{0.56,0.35,0.01}{#1}}
\newcommand{\PreprocessorTok}[1]{\textcolor[rgb]{0.56,0.35,0.01}{\textit{#1}}}
\newcommand{\RegionMarkerTok}[1]{#1}
\newcommand{\SpecialCharTok}[1]{\textcolor[rgb]{0.81,0.36,0.00}{\textbf{#1}}}
\newcommand{\SpecialStringTok}[1]{\textcolor[rgb]{0.31,0.60,0.02}{#1}}
\newcommand{\StringTok}[1]{\textcolor[rgb]{0.31,0.60,0.02}{#1}}
\newcommand{\VariableTok}[1]{\textcolor[rgb]{0.00,0.00,0.00}{#1}}
\newcommand{\VerbatimStringTok}[1]{\textcolor[rgb]{0.31,0.60,0.02}{#1}}
\newcommand{\WarningTok}[1]{\textcolor[rgb]{0.56,0.35,0.01}{\textbf{\textit{#1}}}}
\usepackage{longtable,booktabs,array}
\usepackage{calc} % for calculating minipage widths
% Correct order of tables after \paragraph or \subparagraph
\usepackage{etoolbox}
\makeatletter
\patchcmd\longtable{\par}{\if@noskipsec\mbox{}\fi\par}{}{}
\makeatother
% Allow footnotes in longtable head/foot
\IfFileExists{footnotehyper.sty}{\usepackage{footnotehyper}}{\usepackage{footnote}}
\makesavenoteenv{longtable}
\usepackage{graphicx}
\makeatletter
\newsavebox\pandoc@box
\newcommand*\pandocbounded[1]{% scales image to fit in text height/width
  \sbox\pandoc@box{#1}%
  \Gscale@div\@tempa{\textheight}{\dimexpr\ht\pandoc@box+\dp\pandoc@box\relax}%
  \Gscale@div\@tempb{\linewidth}{\wd\pandoc@box}%
  \ifdim\@tempb\p@<\@tempa\p@\let\@tempa\@tempb\fi% select the smaller of both
  \ifdim\@tempa\p@<\p@\scalebox{\@tempa}{\usebox\pandoc@box}%
  \else\usebox{\pandoc@box}%
  \fi%
}
% Set default figure placement to htbp
\def\fps@figure{htbp}
\makeatother
\setlength{\emergencystretch}{3em} % prevent overfull lines
\providecommand{\tightlist}{%
  \setlength{\itemsep}{0pt}\setlength{\parskip}{0pt}}
\usepackage{bookmark}
\IfFileExists{xurl.sty}{\usepackage{xurl}}{} % add URL line breaks if available
\urlstyle{same}
\hypersetup{
  pdftitle={Mini-Research Project},
  hidelinks,
  pdfcreator={LaTeX via pandoc}}

\title{Mini-Research Project}
\author{}
\date{\vspace{-2.5em}}

\begin{document}
\maketitle

\section{Antibiotic Resistance
Patterns}\label{antibiotic-resistance-patterns}

\subsubsection{Team members: Daryna Horetska, Yaryna Pechenenko, Andrii
Hovorov}\label{team-members-daryna-horetska-yaryna-pechenenko-andrii-hovorov}

\textbf{Dataset}: Multi-Resistance Antibiotic Susceptibility

10 710 records that contains bacterial isolates, associated risk
factors, and antibiotic susceptibility results per patient.

\url{https://www.kaggle.com/datasets/adilimadeddinehosni/multi-resistance-antibiotic-susceptibility}

\textbf{Our goal}: to analyse bacteria resistance to different
antibiotic depending on age, previous diseases, gender etc.

\subsection{Data description}\label{data-description}

\paragraph{Bacteria Sample
Identifiers}\label{bacteria-sample-identifiers}

\begin{itemize}
\tightlist
\item
  ID
\item
  Souche (Strain) -- bacteria name
\end{itemize}

\paragraph{Patient Demographics}\label{patient-demographics}

\begin{itemize}
\tightlist
\item
  Age (numeric)
\item
  Gender (encoded as M or F)
\item
  Location (US state, district etc.)
\end{itemize}

\paragraph{Clinical Risk Factors}\label{clinical-risk-factors}

\begin{itemize}
\tightlist
\item
  Diabetes
\item
  Hypertension
\item
  Prior hospitalization (within the last 6 months)
\item
  Frequency of recurrent infections
\end{itemize}

\paragraph{Infection Information}\label{infection-information}

\begin{itemize}
\tightlist
\item
  Specific bacterial infection identifying the illness
\item
  Antibiotic Susceptibility Test Results
\end{itemize}

Scope: Tested against 15 antibiotics (grouped by antibiotic family)

Result Values: Resistant, Susceptible, Intermediate, or Missing data

\paragraph{Metadata}\label{metadata}

\begin{itemize}
\tightlist
\item
  Date of sample collection
\item
  Free-text comments (in various languages)
\end{itemize}

All the data is raw and dirty, therefore needs data cleaning before
conduction of any analysis.

\subsubsection{Set up necessary
libraries}\label{set-up-necessary-libraries}

First, let us set up all the necessary libraries for our research in
purpose of data visualization, descriptive analysis, text cleaning,
separating columns, providing Breslow-Day test.

\begin{Shaded}
\begin{Highlighting}[]
\FunctionTok{library}\NormalTok{(tidyr)}
\FunctionTok{library}\NormalTok{(dplyr)}
\end{Highlighting}
\end{Shaded}

\begin{verbatim}
## 
## Attaching package: 'dplyr'
\end{verbatim}

\begin{verbatim}
## The following objects are masked from 'package:stats':
## 
##     filter, lag
\end{verbatim}

\begin{verbatim}
## The following objects are masked from 'package:base':
## 
##     intersect, setdiff, setequal, union
\end{verbatim}

\begin{Shaded}
\begin{Highlighting}[]
\FunctionTok{library}\NormalTok{(stringr)}
\FunctionTok{library}\NormalTok{(DescTools)}
\end{Highlighting}
\end{Shaded}

\begin{verbatim}
## Warning: package 'DescTools' was built under R version 4.5.2
\end{verbatim}

\begin{Shaded}
\begin{Highlighting}[]
\FunctionTok{library}\NormalTok{(ggplot2)}
\end{Highlighting}
\end{Shaded}

\subsection{Data cleaning}\label{data-cleaning}

Cleaning data: - replace unknown data by NA, clean it - separate age and
gender - extract clean (w/o ID) bacteria name - unified each name of
bacteria and states of resistance

\begin{Shaded}
\begin{Highlighting}[]
\NormalTok{dataset\_readed }\OtherTok{\textless{}{-}} \FunctionTok{read.csv}\NormalTok{(}\StringTok{"Bacteria\_dataset\_Multiresictance.csv"}\NormalTok{)}

\NormalTok{dataset\_readed[dataset\_readed }\SpecialCharTok{==} \StringTok{"?"}\NormalTok{] }\OtherTok{\textless{}{-}} \ConstantTok{NA}
\NormalTok{dataset\_readed[dataset\_readed }\SpecialCharTok{==} \StringTok{"missing"}\NormalTok{] }\OtherTok{\textless{}{-}} \ConstantTok{NA}
\NormalTok{dataset\_readed[dataset\_readed }\SpecialCharTok{==} \StringTok{"unknown"}\NormalTok{] }\OtherTok{\textless{}{-}} \ConstantTok{NA}
\NormalTok{dataset\_readed[dataset\_readed }\SpecialCharTok{==} \StringTok{"error"}\NormalTok{] }\OtherTok{\textless{}{-}} \ConstantTok{NA}
\NormalTok{dataset\_readed[dataset\_readed }\SpecialCharTok{==} \StringTok{"???"}\NormalTok{] }\OtherTok{\textless{}{-}} \ConstantTok{NA}
\NormalTok{dataset\_readed[dataset\_readed }\SpecialCharTok{==} \StringTok{"??"}\NormalTok{] }\OtherTok{\textless{}{-}} \ConstantTok{NA}
\NormalTok{dataset\_readed[dataset\_readed }\SpecialCharTok{==} \StringTok{""}\NormalTok{] }\OtherTok{\textless{}{-}} \ConstantTok{NA}
\NormalTok{dataset\_readed[dataset\_readed }\SpecialCharTok{==} \StringTok{"None"}\NormalTok{] }\OtherTok{\textless{}{-}} \ConstantTok{NA}
\NormalTok{dataset\_readed[dataset\_readed }\SpecialCharTok{==} \StringTok{"null"}\NormalTok{] }\OtherTok{\textless{}{-}} \ConstantTok{NA}
\NormalTok{dataset\_readed }\OtherTok{\textless{}{-}}\NormalTok{ dataset\_readed }\SpecialCharTok{\%\textgreater{}\%} 
  \FunctionTok{separate}\NormalTok{(Souches, }\AttributeTok{into =} \FunctionTok{c}\NormalTok{(}\StringTok{"Strain\_ID"}\NormalTok{, }\StringTok{"souche\_description"}\NormalTok{), }
           \AttributeTok{sep =} \StringTok{" "}\NormalTok{, }\AttributeTok{extra =} \StringTok{"merge"}\NormalTok{, }\AttributeTok{fill =} \StringTok{"right"}\NormalTok{)}
\NormalTok{dataset\_readed}\SpecialCharTok{$}\NormalTok{souche\_description }\OtherTok{\textless{}{-}} \FunctionTok{tolower}\NormalTok{(dataset\_readed}\SpecialCharTok{$}\NormalTok{souche\_description)}
\NormalTok{dataset\_readed}\SpecialCharTok{$}\NormalTok{souche\_description }\OtherTok{\textless{}{-}} \FunctionTok{trimws}\NormalTok{(dataset\_readed}\SpecialCharTok{$}\NormalTok{souche\_description)}

\CommentTok{\# Escherichia coli}
\NormalTok{dataset\_readed}\SpecialCharTok{$}\NormalTok{souche\_description[dataset\_readed}\SpecialCharTok{$}\NormalTok{souche\_description }\SpecialCharTok{==} \StringTok{"e.coi"} 
                                  \SpecialCharTok{|}\NormalTok{ dataset\_readed}\SpecialCharTok{$}\NormalTok{souche\_description }\SpecialCharTok{==} \StringTok{"e coli"} 
                                  \SpecialCharTok{|}\NormalTok{ dataset\_readed}\SpecialCharTok{$}\NormalTok{souche\_description }\SpecialCharTok{==} \StringTok{"e.cli"} 
                                  \SpecialCharTok{|}\NormalTok{ dataset\_readed}\SpecialCharTok{$}\NormalTok{souche\_description }\SpecialCharTok{==} 
                                    \StringTok{"e. coli"}\NormalTok{] }\OtherTok{\textless{}{-}} \StringTok{"escherichia coli"}

\CommentTok{\# Enterobacteria spp.}
\NormalTok{dataset\_readed}\SpecialCharTok{$}\NormalTok{souche\_description[dataset\_readed}\SpecialCharTok{$}\NormalTok{souche\_description }\SpecialCharTok{==} 
                                    \StringTok{"spp. enteobacteria spp."}\NormalTok{] }\OtherTok{\textless{}{-}} \StringTok{"enterobacteria spp."}
\NormalTok{dataset\_readed}\SpecialCharTok{$}\NormalTok{souche\_description[dataset\_readed}\SpecialCharTok{$}\NormalTok{souche\_description }\SpecialCharTok{==} 
                                    \StringTok{"enter.bacteria"} 
                                  \SpecialCharTok{|}\NormalTok{ dataset\_readed}\SpecialCharTok{$}\NormalTok{souche\_description }\SpecialCharTok{==} \StringTok{"enteobacteria spp."} 
                                  \SpecialCharTok{|}\NormalTok{ dataset\_readed}\SpecialCharTok{$}\NormalTok{souche\_description }\SpecialCharTok{==} 
                                    \StringTok{"enter.bacteria spp."}\NormalTok{] }\OtherTok{\textless{}{-}} \StringTok{"enterobacteria spp."}

\CommentTok{\# Klebsiella pneumoniae}
\NormalTok{dataset\_readed}\SpecialCharTok{$}\NormalTok{souche\_description[dataset\_readed}\SpecialCharTok{$}\NormalTok{souche\_description }\SpecialCharTok{==} \StringTok{"klbsiella pneumoniae"} 
                                  \SpecialCharTok{|}\NormalTok{dataset\_readed}\SpecialCharTok{$}\NormalTok{souche\_description }\SpecialCharTok{==} 
                                    \StringTok{"klebsie.lla pneumoniae"}\NormalTok{] }\OtherTok{\textless{}{-}} \StringTok{"klebsiella pneumoniae"}

\CommentTok{\# Proteus mirabilis}
\NormalTok{dataset\_readed}\SpecialCharTok{$}\NormalTok{souche\_description[dataset\_readed}\SpecialCharTok{$}\NormalTok{souche\_description }\SpecialCharTok{==} \StringTok{"protus mirabilis"} 
                                  \SpecialCharTok{|}\NormalTok{ dataset\_readed}\SpecialCharTok{$}\NormalTok{souche\_description }\SpecialCharTok{==} \StringTok{"prot.eus mirabilis"} 
                                  \SpecialCharTok{|}\NormalTok{ dataset\_readed}\SpecialCharTok{$}\NormalTok{souche\_description }\SpecialCharTok{==} 
                                    \StringTok{"proeus mirabilis"}\NormalTok{] }\OtherTok{\textless{}{-}} \StringTok{"proteus mirabilis"}

\NormalTok{antibiotics }\OtherTok{\textless{}{-}} \FunctionTok{c}\NormalTok{(}\StringTok{"AMX.AMP"}\NormalTok{, }\StringTok{"AMC"}\NormalTok{, }\StringTok{"CZ"}\NormalTok{, }\StringTok{"FOX"}\NormalTok{, }\StringTok{"CTX.CRO"}\NormalTok{, }\StringTok{"IPM"}\NormalTok{, }\StringTok{"GEN"}\NormalTok{, }\StringTok{"AN"}\NormalTok{, }
                 \StringTok{"Acide.nalidixique"}\NormalTok{, }\StringTok{"ofx"}\NormalTok{, }\StringTok{"CIP"}\NormalTok{, }\StringTok{"C"}\NormalTok{, }
                 \StringTok{"Co.trimoxazole"}\NormalTok{, }\StringTok{"Furanes"}\NormalTok{, }\StringTok{"colistine"}\NormalTok{)}

\NormalTok{dataset\_readed }\OtherTok{\textless{}{-}} \FunctionTok{na.omit}\NormalTok{(dataset\_readed)}

\NormalTok{dataset\_readed[antibiotics] }\OtherTok{\textless{}{-}} \FunctionTok{lapply}\NormalTok{(dataset\_readed[antibiotics], }\ControlFlowTok{function}\NormalTok{(x) \{}
\NormalTok{  x[x }\SpecialCharTok{==} \StringTok{"r"}\NormalTok{] }\OtherTok{\textless{}{-}} \StringTok{"R"}
\NormalTok{  x[x }\SpecialCharTok{==} \StringTok{"intermediate"}\NormalTok{] }\OtherTok{\textless{}{-}} \StringTok{"R"}
\NormalTok{  x[x }\SpecialCharTok{==} \StringTok{"Intermediate"}\NormalTok{] }\OtherTok{\textless{}{-}} \StringTok{"R"}
\NormalTok{  x[x }\SpecialCharTok{==} \StringTok{"i"}\NormalTok{] }\OtherTok{\textless{}{-}} \StringTok{"R"}
\NormalTok{  x[x }\SpecialCharTok{==} \StringTok{"s"}\NormalTok{] }\OtherTok{\textless{}{-}} \StringTok{"S"}
\NormalTok{  x}
\NormalTok{\})}

\NormalTok{dataset\_readed }\OtherTok{\textless{}{-}}\NormalTok{ dataset\_readed }\SpecialCharTok{\%\textgreater{}\%}
  \FunctionTok{separate}\NormalTok{(}\StringTok{\textasciigrave{}}\AttributeTok{age.gender}\StringTok{\textasciigrave{}}\NormalTok{, }\AttributeTok{into =} \FunctionTok{c}\NormalTok{(}\StringTok{"age"}\NormalTok{, }\StringTok{"gender"}\NormalTok{), }\AttributeTok{sep =} \StringTok{"/"}\NormalTok{, }\AttributeTok{remove =} \ConstantTok{FALSE}\NormalTok{)}
\end{Highlighting}
\end{Shaded}

After cleaning our data 8281 records remain.

\subsection{Descriptive analysis}\label{descriptive-analysis}

\subsubsection{Initial descriptive analysis of ages and infection
frequency}\label{initial-descriptive-analysis-of-ages-and-infection-frequency}

\begin{Shaded}
\begin{Highlighting}[]
\NormalTok{dataset\_readed}\SpecialCharTok{$}\NormalTok{age }\OtherTok{\textless{}{-}} \FunctionTok{as.numeric}\NormalTok{(dataset\_readed}\SpecialCharTok{$}\NormalTok{age)}

\FunctionTok{cat}\NormalTok{(}\StringTok{"Descriptive analysis of ages}\SpecialCharTok{\textbackslash{}n}\StringTok{"}\NormalTok{)}
\end{Highlighting}
\end{Shaded}

\begin{verbatim}
## Descriptive analysis of ages
\end{verbatim}

\begin{Shaded}
\begin{Highlighting}[]
\FunctionTok{summary}\NormalTok{(dataset\_readed}\SpecialCharTok{$}\NormalTok{age)}
\end{Highlighting}
\end{Shaded}

\begin{verbatim}
##    Min. 1st Qu.  Median    Mean 3rd Qu.    Max. 
##    0.00   25.00   45.00   45.67   67.00   90.00
\end{verbatim}

\begin{Shaded}
\begin{Highlighting}[]
\FunctionTok{Mode}\NormalTok{(dataset\_readed}\SpecialCharTok{$}\NormalTok{age)}
\end{Highlighting}
\end{Shaded}

\begin{verbatim}
## [1] 50
## attr(,"freq")
## [1] 167
\end{verbatim}

\begin{Shaded}
\begin{Highlighting}[]
\FunctionTok{Kurt}\NormalTok{(dataset\_readed}\SpecialCharTok{$}\NormalTok{age)}
\end{Highlighting}
\end{Shaded}

\begin{verbatim}
## [1] -1.050738
\end{verbatim}

\begin{Shaded}
\begin{Highlighting}[]
\FunctionTok{Skew}\NormalTok{(dataset\_readed}\SpecialCharTok{$}\NormalTok{age)}
\end{Highlighting}
\end{Shaded}

\begin{verbatim}
## [1] 0.02307794
\end{verbatim}

We can observe that in our data there is wide spread of age groups (from
infants to elderly persons). The mode age is 50 (that contains 167
times). Therefore, the age is not heavily skewed toward young people.

As the skewness is close to 0 so age distribution is almost symmetrical
so we have nearly equal number of younger and older patients.

Kurtosis is negative (Platykurtic) so the distribution has thick tails.

\subsubsection{Frequencies of unique bacteria
souches}\label{frequencies-of-unique-bacteria-souches}

We have 9 different types of bacteria.

\begin{Shaded}
\begin{Highlighting}[]
\FunctionTok{n\_distinct}\NormalTok{(dataset\_readed}\SpecialCharTok{$}\NormalTok{souche\_description)}
\end{Highlighting}
\end{Shaded}

\begin{verbatim}
## [1] 9
\end{verbatim}

There are frequencies of each bacteria we have so in the further part we
will base our test only on the most frequent types:

\begin{Shaded}
\begin{Highlighting}[]
\NormalTok{dataset\_readed }\SpecialCharTok{\%\textgreater{}\%}
  \FunctionTok{count}\NormalTok{(souche\_description, }\AttributeTok{sort =} \ConstantTok{TRUE}\NormalTok{) }\SpecialCharTok{\%\textgreater{}\%}
  \FunctionTok{ggplot}\NormalTok{(}\FunctionTok{aes}\NormalTok{(}\AttributeTok{x =} \FunctionTok{reorder}\NormalTok{(souche\_description, n), }\AttributeTok{y =}\NormalTok{ n)) }\SpecialCharTok{+}
  \FunctionTok{geom\_col}\NormalTok{() }\SpecialCharTok{+}
  \FunctionTok{coord\_flip}\NormalTok{() }\SpecialCharTok{+}
  \FunctionTok{labs}\NormalTok{(}\AttributeTok{title =} \StringTok{"Frequency of Bacteria Species"}\NormalTok{,}
    \AttributeTok{x =} \StringTok{"Bacteria"}\NormalTok{,}
    \AttributeTok{y =} \StringTok{"Amount"}\NormalTok{) }\SpecialCharTok{+} \FunctionTok{theme\_minimal}\NormalTok{()}
\end{Highlighting}
\end{Shaded}

\pandocbounded{\includegraphics[keepaspectratio]{interim_new_files/figure-latex/unnamed-chunk-5-1.pdf}}

\subsubsection{Distribution of total
age}\label{distribution-of-total-age}

As analysed before, we can observe that out distribution is quite
symmetrical and platykurtic, indeed. For our further ideas on influence
of diabetes it would be crucial to know that age distribution of groups
of patients that have either resistant or susceptible infection. As
diabetes is more common among elderly people, if one our groups would
have more elder people we wouldn't be able to make conclusions on
dependence between these two metrics (diabetes and resistance).

\begin{Shaded}
\begin{Highlighting}[]
\NormalTok{age }\OtherTok{\textless{}{-}}\NormalTok{ dataset\_readed}\SpecialCharTok{$}\NormalTok{age}
\FunctionTok{hist}\NormalTok{(age,}\AttributeTok{breaks =} \FunctionTok{seq}\NormalTok{(}\FunctionTok{min}\NormalTok{(age,}\AttributeTok{na.rm =} \ConstantTok{TRUE}\NormalTok{), }\FunctionTok{max}\NormalTok{(age, }\AttributeTok{na.rm =} \ConstantTok{TRUE}\NormalTok{), }\AttributeTok{by =} \DecValTok{2}\NormalTok{), }
     \AttributeTok{col =} \StringTok{"skyblue"}\NormalTok{,}
     \AttributeTok{border =} \StringTok{"white"}\NormalTok{,}
     \AttributeTok{main =} \StringTok{"Distribution of patient ages"}\NormalTok{,}
     \AttributeTok{xlab =} \StringTok{"Age"}\NormalTok{,}
     \AttributeTok{ylab =} \StringTok{"Number of patients"}\NormalTok{)}
\end{Highlighting}
\end{Shaded}

\pandocbounded{\includegraphics[keepaspectratio]{interim_new_files/figure-latex/unnamed-chunk-6-1.pdf}}

\subsubsection{Distributions of age by two groups: resistant,
susceptible}\label{distributions-of-age-by-two-groups-resistant-susceptible}

We divided patients into 2 groups: resistant (when patient is resistant
to \textgreater{} 5 types of antibiotics) and susceptible otherwise

\begin{Shaded}
\begin{Highlighting}[]
\NormalTok{abx\_data }\OtherTok{\textless{}{-}}\NormalTok{ dataset\_readed[, antibiotics]}
\NormalTok{dataset\_readed}\SpecialCharTok{$}\NormalTok{R\_count }\OtherTok{\textless{}{-}} \FunctionTok{rowSums}\NormalTok{(abx\_data }\SpecialCharTok{==} \StringTok{"R"}\NormalTok{, }\AttributeTok{na.rm =} \ConstantTok{TRUE}\NormalTok{)}
\NormalTok{dataset\_readed}\SpecialCharTok{$}\NormalTok{Resistance\_Status }\OtherTok{\textless{}{-}} \FunctionTok{ifelse}\NormalTok{(dataset\_readed}\SpecialCharTok{$}\NormalTok{R\_count }\SpecialCharTok{\textgreater{}} \DecValTok{5}\NormalTok{, }\StringTok{\textquotesingle{}Resistant\textquotesingle{}}\NormalTok{, }\StringTok{\textquotesingle{}Susceptible\textquotesingle{}}\NormalTok{)}
\NormalTok{group\_r }\OtherTok{\textless{}{-}}\NormalTok{ dataset\_readed}\SpecialCharTok{$}\NormalTok{age[dataset\_readed}\SpecialCharTok{$}\NormalTok{Resistance\_Status }\SpecialCharTok{==} \StringTok{\textquotesingle{}Resistant\textquotesingle{}}\NormalTok{]}
\NormalTok{group\_s }\OtherTok{\textless{}{-}}\NormalTok{ dataset\_readed}\SpecialCharTok{$}\NormalTok{age[dataset\_readed}\SpecialCharTok{$}\NormalTok{Resistance\_Status }\SpecialCharTok{==} \StringTok{\textquotesingle{}Susceptible\textquotesingle{}}\NormalTok{]}
\end{Highlighting}
\end{Shaded}

\begin{Shaded}
\begin{Highlighting}[]
\FunctionTok{par}\NormalTok{(}\AttributeTok{mfrow =} \FunctionTok{c}\NormalTok{(}\DecValTok{1}\NormalTok{, }\DecValTok{2}\NormalTok{))}
\FunctionTok{hist}\NormalTok{(group\_r,}
     \AttributeTok{col =} \StringTok{"pink"}\NormalTok{,}
     \AttributeTok{border =} \StringTok{\textquotesingle{}magenta\textquotesingle{}}\NormalTok{,}
     \AttributeTok{main =} \StringTok{"Ages with resistant infection"}\NormalTok{,}
     \AttributeTok{xlab =} \StringTok{"Age"}\NormalTok{,}
     \AttributeTok{ylab =} \StringTok{"Number of patients"}\NormalTok{)}

\FunctionTok{hist}\NormalTok{(group\_s,}
     \AttributeTok{col =} \StringTok{"cyan"}\NormalTok{,}
     \AttributeTok{border =} \StringTok{"blue"}\NormalTok{,}
     \AttributeTok{main =} \StringTok{"Ages with susceptible infection"}\NormalTok{,}
     \AttributeTok{xlab =} \StringTok{"Age"}\NormalTok{,}
     \AttributeTok{ylab =} \StringTok{"Number of patients"}\NormalTok{)}
\end{Highlighting}
\end{Shaded}

\pandocbounded{\includegraphics[keepaspectratio]{interim_new_files/figure-latex/unnamed-chunk-8-1.pdf}}

\begin{Shaded}
\begin{Highlighting}[]
\FunctionTok{par}\NormalTok{(}\AttributeTok{mfrow =} \FunctionTok{c}\NormalTok{(}\DecValTok{1}\NormalTok{, }\DecValTok{1}\NormalTok{))}
\end{Highlighting}
\end{Shaded}

\subsubsection{Central tendencies of age by two groups: resistant,
susceptible}\label{central-tendencies-of-age-by-two-groups-resistant-susceptible}

\begin{Shaded}
\begin{Highlighting}[]
\CommentTok{\# 2. Create the Visualization using your dataframe \textquotesingle{}dataset\_readed\textquotesingle{}}
\FunctionTok{ggplot}\NormalTok{(dataset\_readed, }\FunctionTok{aes}\NormalTok{(}\AttributeTok{x =}\NormalTok{ Resistance\_Status, }\AttributeTok{y =}\NormalTok{ age, }\AttributeTok{fill =}\NormalTok{ Resistance\_Status)) }\SpecialCharTok{+}
  \FunctionTok{geom\_violin}\NormalTok{(}\AttributeTok{trim =} \ConstantTok{FALSE}\NormalTok{, }\AttributeTok{alpha =} \FloatTok{0.7}\NormalTok{) }\SpecialCharTok{+}
  \FunctionTok{geom\_boxplot}\NormalTok{(}\AttributeTok{width =} \FloatTok{0.1}\NormalTok{, }\AttributeTok{fill =} \StringTok{"white"}\NormalTok{, }\AttributeTok{outlier.shape =} \ConstantTok{NA}\NormalTok{) }\SpecialCharTok{+}
  \FunctionTok{theme\_minimal}\NormalTok{()}
\end{Highlighting}
\end{Shaded}

\pandocbounded{\includegraphics[keepaspectratio]{interim_new_files/figure-latex/unnamed-chunk-9-1.pdf}}

From this violin plot we can assume that age distributions of
susceptible and resistant groups are similar so we will test this
assumption using Kolmogorov-Smirnov test.

\subsubsection{Diabetes vs.~Resistance Bar
Chart}\label{diabetes-vs.-resistance-bar-chart}

\begin{Shaded}
\begin{Highlighting}[]
\FunctionTok{ggplot}\NormalTok{(dataset\_readed, }\FunctionTok{aes}\NormalTok{(}\AttributeTok{x =}\NormalTok{ Diabetes, }\AttributeTok{fill =}\NormalTok{ Resistance\_Status)) }\SpecialCharTok{+}
  \FunctionTok{geom\_bar}\NormalTok{(}\AttributeTok{position =} \StringTok{"fill"}\NormalTok{, }\AttributeTok{width =} \FloatTok{0.6}\NormalTok{) }\SpecialCharTok{+}
  
  \FunctionTok{scale\_y\_continuous}\NormalTok{(}\AttributeTok{labels =}\NormalTok{ scales}\SpecialCharTok{::}\NormalTok{percent) }\SpecialCharTok{+}
  
  \FunctionTok{scale\_fill\_manual}\NormalTok{(}\AttributeTok{values =} \FunctionTok{c}\NormalTok{(}\StringTok{"Susceptible"} \OtherTok{=} \StringTok{"lightgreen"}\NormalTok{, }\StringTok{"Resistant"} \OtherTok{=} \StringTok{"orange"}\NormalTok{)) }\SpecialCharTok{+}
  
  \FunctionTok{labs}\NormalTok{(}\AttributeTok{title =} \StringTok{"Impact of Diabetes on Antibiotic Resistance"}\NormalTok{,}
       \AttributeTok{subtitle =} \StringTok{"Proportion of Resistant Infections by Diabetes Status"}\NormalTok{,}
       \AttributeTok{x =} \StringTok{"Diabetes Status"}\NormalTok{,}
       \AttributeTok{y =} \StringTok{"Percentage"}\NormalTok{,}
       \AttributeTok{fill =} \StringTok{"Infection Status"}\NormalTok{) }\SpecialCharTok{+}
  
  \FunctionTok{theme\_minimal}\NormalTok{()}
\end{Highlighting}
\end{Shaded}

\pandocbounded{\includegraphics[keepaspectratio]{interim_new_files/figure-latex/unnamed-chunk-10-1.pdf}}

From this plot we observe that the proportion of people with resistant
and susceptible infection almost equal among diabetes so we are going to
test it and dive deeper into dependences for some types of bacteria.

\subsubsection{Heatmap of diabetes impact on resistance on different
souches}\label{heatmap-of-diabetes-impact-on-resistance-on-different-souches}

\begin{Shaded}
\begin{Highlighting}[]
\NormalTok{heatmap\_data }\OtherTok{\textless{}{-}}\NormalTok{ dataset\_readed }\SpecialCharTok{\%\textgreater{}\%}
  \FunctionTok{group\_by}\NormalTok{(souche\_description, Diabetes) }\SpecialCharTok{\%\textgreater{}\%}
  \FunctionTok{summarise}\NormalTok{(}
    \AttributeTok{Total\_Cases =} \FunctionTok{n}\NormalTok{(),}
    \AttributeTok{Resistant\_Count =} \FunctionTok{sum}\NormalTok{(Resistance\_Status }\SpecialCharTok{==} \StringTok{"Resistant"}\NormalTok{),}
    \AttributeTok{Pct\_Resistant =}\NormalTok{ Resistant\_Count }\SpecialCharTok{/}\NormalTok{ Total\_Cases,}
    \AttributeTok{.groups =} \StringTok{"drop"}
\NormalTok{  )}

\FunctionTok{ggplot}\NormalTok{(heatmap\_data, }\FunctionTok{aes}\NormalTok{(}\AttributeTok{x =}\NormalTok{ Diabetes, }\AttributeTok{y =}\NormalTok{ souche\_description, }\AttributeTok{fill =}\NormalTok{ Pct\_Resistant)) }\SpecialCharTok{+}
  \FunctionTok{geom\_tile}\NormalTok{(}\AttributeTok{color =} \StringTok{"white"}\NormalTok{) }\SpecialCharTok{+}
  \FunctionTok{geom\_text}\NormalTok{(}\FunctionTok{aes}\NormalTok{(}\AttributeTok{label =}\NormalTok{ scales}\SpecialCharTok{::}\FunctionTok{percent}\NormalTok{(Pct\_Resistant, }\AttributeTok{accuracy =} \DecValTok{1}\NormalTok{)), }\AttributeTok{color =} \StringTok{"black"}\NormalTok{, }\AttributeTok{size =} \DecValTok{3}\NormalTok{) }\SpecialCharTok{+}
  \FunctionTok{scale\_fill\_gradient}\NormalTok{(}\AttributeTok{low =} \StringTok{"\#fff7bc"}\NormalTok{, }\AttributeTok{high =} \StringTok{"\#d95f0e"}\NormalTok{, }\AttributeTok{labels =}\NormalTok{ scales}\SpecialCharTok{::}\NormalTok{percent) }\SpecialCharTok{+}
  \FunctionTok{labs}\NormalTok{(}
    \AttributeTok{title =} \StringTok{"Antibiotic Resistance Heatmap"}\NormalTok{,}
    \AttributeTok{subtitle =} \StringTok{"Percentage of Resistant Cases by Bacteria and Diabetes Status"}\NormalTok{,}
    \AttributeTok{x =} \StringTok{"Diabetes Status"}\NormalTok{,}
    \AttributeTok{y =} \StringTok{"Bacteria Type (Souche)"}\NormalTok{,}
    \AttributeTok{fill =} \StringTok{"Resistance \%"}
\NormalTok{  ) }\SpecialCharTok{+}
  \FunctionTok{theme\_minimal}\NormalTok{() }\SpecialCharTok{+} 
  \FunctionTok{theme}\NormalTok{(}
    \AttributeTok{panel.grid =} \FunctionTok{element\_blank}\NormalTok{()}
\NormalTok{  )}
\end{Highlighting}
\end{Shaded}

\pandocbounded{\includegraphics[keepaspectratio]{interim_new_files/figure-latex/unnamed-chunk-11-1.pdf}}

From this heat map we can observe that most types of bacteria have the
same resistance level not depending on diabetes status. But we also
should test it to get more precise result.

\section{Hypotheses testing}\label{hypotheses-testing}

To start with, as one of our hypotheses is based on age metric, it is
natural to test the similarity of resistant and susceptible age group
distributions.

\subsection{Age tendencies}\label{age-tendencies}

First we will compare overall distributions of groups using
Kolmogorov-Smirnov test \(H_0\): resistant and susceptible groups have
the same distribution \(H_1\): resistant and susceptible groups have not
the same distribution

\begin{Shaded}
\begin{Highlighting}[]
\FunctionTok{ks.test}\NormalTok{(}\FunctionTok{jitter}\NormalTok{(group\_r), }\FunctionTok{jitter}\NormalTok{(group\_s))}
\end{Highlighting}
\end{Shaded}

\begin{verbatim}
## 
##  Asymptotic two-sample Kolmogorov-Smirnov test
## 
## data:  jitter(group_r) and jitter(group_s)
## D = 0.01969, p-value = 0.4108
## alternative hypothesis: two-sided
\end{verbatim}

\begin{Shaded}
\begin{Highlighting}[]
\FunctionTok{plot}\NormalTok{(}\FunctionTok{ecdf}\NormalTok{(group\_s), }\AttributeTok{col=}\StringTok{"cyan"}\NormalTok{, }\AttributeTok{main=}\StringTok{"ECDFs comparison"}\NormalTok{)}
\NormalTok{s\_range }\OtherTok{\textless{}{-}} \FunctionTok{seq}\NormalTok{(}\FunctionTok{min}\NormalTok{(group\_s), }\FunctionTok{max}\NormalTok{(group\_s), }\AttributeTok{length.out =} \DecValTok{500}\NormalTok{)}
\FunctionTok{lines}\NormalTok{(s\_range, }
      \FunctionTok{ecdf}\NormalTok{(group\_r)(s\_range), }
      \AttributeTok{col =} \StringTok{"magenta"}\NormalTok{, }
      \AttributeTok{lwd =} \DecValTok{2}\NormalTok{)}
\end{Highlighting}
\end{Shaded}

\pandocbounded{\includegraphics[keepaspectratio]{interim_new_files/figure-latex/unnamed-chunk-13-1.pdf}}

Difference between distribution of resistant group and susceptible are
pretty small, p-value 0,4808 is greater than 0,05 so we fail to reject
\(H_0\) Therefore the distributions of resistant and susceptible groups
are equal. Kolmogorov-Smirnov test for whether distributions of two
groups of patients are equal. -\textgreater{} \textbf{Not Rejected}

Therefore, we can conclude that the age does not influence the
resistance of bacterial infection.

\subsection{Diabetes impact on
resistance}\label{diabetes-impact-on-resistance}

As we have some info on risk factors in our dataset, we decided to test
whether some of them influence the resistance.

Our aim here was firstly to test whether there is connection between
having diabetes and more resistant infection at all. And then we can
further test if the type of bacteria has an impact on that connection,
meaning impact of diabetes on resistance depends on type of infection
for which we test the resistance. Therefore, we are performing the
following:

\begin{enumerate}
\def\labelenumi{\arabic{enumi}.}
\item
  Construct a contingency table for Chi-squared test for independence
\item
  Perform Chi-Squared test for independence (of diabetes and resistance
  factor)
\end{enumerate}

Though by itself, long term diseases like diabetes or hypertension do
not directly influence resistance of bacterial infections to
antibiotics, they do influence the frequency of illness, which then
influences antibiotic consumption and that influences bacterial
adaptability (which is extremely fast) and poses a resistance.

\(H_0\): people who have diabetes are not more prone to have resistance
to antibiotics \(H_1\): people who have diabetes are more prone to have
resistance to antibiotics

\begin{Shaded}
\begin{Highlighting}[]
\NormalTok{diabetic\_r }\OtherTok{\textless{}{-}}\NormalTok{ dataset\_readed}\SpecialCharTok{$}\NormalTok{ID[dataset\_readed}\SpecialCharTok{$}\NormalTok{Resistance\_Status }\SpecialCharTok{==} 
                                  \StringTok{\textquotesingle{}Resistant\textquotesingle{}}\SpecialCharTok{\&}\NormalTok{ dataset\_readed}\SpecialCharTok{$}\NormalTok{Diabetes }\SpecialCharTok{==} \StringTok{"True"}\NormalTok{ ]}
\NormalTok{diabetic\_s }\OtherTok{\textless{}{-}}\NormalTok{ dataset\_readed}\SpecialCharTok{$}\NormalTok{ID[dataset\_readed}\SpecialCharTok{$}\NormalTok{Resistance\_Status }\SpecialCharTok{==} 
                                  \StringTok{\textquotesingle{}Susceptible\textquotesingle{}} \SpecialCharTok{\&}\NormalTok{ dataset\_readed}\SpecialCharTok{$}\NormalTok{Diabetes }\SpecialCharTok{==} \StringTok{"True"}\NormalTok{ ]}

\NormalTok{nondiabetic\_r }\OtherTok{\textless{}{-}}\NormalTok{ dataset\_readed}\SpecialCharTok{$}\NormalTok{ID[dataset\_readed}\SpecialCharTok{$}\NormalTok{Resistance\_Status }\SpecialCharTok{==} 
                                     \StringTok{\textquotesingle{}Resistant\textquotesingle{}}\SpecialCharTok{\&}\NormalTok{ dataset\_readed}\SpecialCharTok{$}\NormalTok{Diabetes }\SpecialCharTok{==} \StringTok{"No"}\NormalTok{ ]}
\NormalTok{nondiabetic\_s }\OtherTok{\textless{}{-}}\NormalTok{  dataset\_readed}\SpecialCharTok{$}\NormalTok{ID[dataset\_readed}\SpecialCharTok{$}\NormalTok{Resistance\_Status }\SpecialCharTok{==} 
                                      \StringTok{\textquotesingle{}Susceptible\textquotesingle{}}\SpecialCharTok{\&}\NormalTok{ dataset\_readed}\SpecialCharTok{$}\NormalTok{Diabetes }\SpecialCharTok{==} \StringTok{"No"}\NormalTok{ ]}
\end{Highlighting}
\end{Shaded}

\begin{Shaded}
\begin{Highlighting}[]
\NormalTok{contingency\_table }\OtherTok{\textless{}{-}} \FunctionTok{table}\NormalTok{(dataset\_readed}\SpecialCharTok{$}\NormalTok{Diabetes, dataset\_readed}\SpecialCharTok{$}\NormalTok{Resistance\_Status)}

\NormalTok{contingency\_table\_marings }\OtherTok{\textless{}{-}} \FunctionTok{addmargins}\NormalTok{(contingency\_table)}

\FunctionTok{print}\NormalTok{(contingency\_table\_marings)}
\end{Highlighting}
\end{Shaded}

\begin{verbatim}
##       
##        Resistant Susceptible  Sum
##   No        2812        3756 6568
##   True       751         962 1713
##   Sum       3563        4718 8281
\end{verbatim}

\begin{Shaded}
\begin{Highlighting}[]
\FunctionTok{chisq.test}\NormalTok{(contingency\_table)}
\end{Highlighting}
\end{Shaded}

\begin{verbatim}
## 
##  Pearson's Chi-squared test with Yates' continuity correction
## 
## data:  contingency_table
## X-squared = 0.54406, df = 1, p-value = 0.4608
\end{verbatim}

Here we can see, that using Chi-Squared Test for independence the
diabetes seems to have no statistically significant impact on resistance
of bacterial infection But now we will dive deeper. Continue to the
previous, we want to check if diabetes influence depends on bacteria
type. We will focus on top3 bacteria infections from dataset, as they
have the most datapoints, and may pose some influence of diabetes on the
resistance \(H_0\): the effect of diabetes on resistance does not depend
on the type of bacteria \(H_1\): the effect of diabetes on resistance
does depend on the type of bacteria

\subsubsection{Diabetes impact depending on type of
bacteria}\label{diabetes-impact-depending-on-type-of-bacteria}

We are now going to construct strata tables for top 3 most frequent
bacteria types (\textbf{Escherichia coli, Enterobacteria spp., Proteus
Mirabilis}) and check if there are impacts of diabetes in these strata.

We see that in fact, using Breslow-Day test, diabetes influence does
depend on bacteria type, as the p-value is smal and we reject null
hypothesis of equals Odds Ratios (look into conclusions for more
detailed explanation)! We can further investigate our discrepnacy with
initial heatmap from data visualizations and the outcome of this test

Therefore we used Breslow-Day Test on Homogeneity of Odds Ratios

\begin{enumerate}
\def\labelenumi{\arabic{enumi}.}
\item
  We first construct three tables for each strata(type of bacteria). The
  table is similar to the one previous test (diabetes/resistance)
\item
  For each strata we calculate Odds Ratio (OR)

  \begin{longtable}[]{@{}lll@{}}
  \toprule\noalign{}
  & Resistant & Susceptible \\
  \midrule\noalign{}
  \endhead
  \bottomrule\noalign{}
  \endlastfoot
  Diabetic & a & b \\
  Non-Diabetic & c & d \\
  \end{longtable}

  \[
  OR = \dfrac{a}{c}\cdot\dfrac{d}{b}
  \]
\end{enumerate}

So if the ratio of diabetic and non diabetic having resistance is the
same as those of susceptible, then OR should be close to 1.

In order to check if the type of bacteria doesn't influence the diabetes
factor on resistance, we should compare OR of all strata (3 types of
bacteria)

\(H_0: \quad OR_1 = OR_2 = OR_3\) , bacteria type does not affect
diabetes factor on resistance

\(H_1: \quad OR_1\ne OR_2 \ne OR_3\) , bacteria type does affect
diabetes factor on resistance

\begin{enumerate}
\def\labelenumi{\arabic{enumi}.}
\setcounter{enumi}{2}
\tightlist
\item
  Then the test uses chi-squared statistic (expected/observed values) to
  test these hypotheses. -\textgreater{} \textbf{Rejected}
\end{enumerate}

\begin{Shaded}
\begin{Highlighting}[]
\NormalTok{top3\_bacteria }\OtherTok{\textless{}{-}} \FunctionTok{c}\NormalTok{(}\StringTok{"escherichia coli"}\NormalTok{, }\StringTok{"enterobacteria spp."}\NormalTok{, }\StringTok{"proteus mirabilis"}\NormalTok{)}
\NormalTok{data\_top3 }\OtherTok{\textless{}{-}}\NormalTok{ dataset\_readed[dataset\_readed}\SpecialCharTok{$}\NormalTok{souche\_description }
                            \SpecialCharTok{\%in\%}\NormalTok{ top3\_bacteria }\SpecialCharTok{\&} \SpecialCharTok{!}\FunctionTok{is.na}\NormalTok{(dataset\_readed}\SpecialCharTok{$}\NormalTok{Diabetes), ]}

\NormalTok{table\_3way }\OtherTok{\textless{}{-}} \FunctionTok{table}\NormalTok{(data\_top3}\SpecialCharTok{$}\NormalTok{Diabetes, data\_top3}\SpecialCharTok{$}\NormalTok{Resistance\_Status, data\_top3}\SpecialCharTok{$}\NormalTok{souche\_description)}
\NormalTok{table\_3way\_safe }\OtherTok{\textless{}{-}} \FunctionTok{array}\NormalTok{(}\FunctionTok{as.numeric}\NormalTok{(table\_3way), }
                         \AttributeTok{dim =} \FunctionTok{dim}\NormalTok{(table\_3way), }
                         \AttributeTok{dimnames =} \FunctionTok{dimnames}\NormalTok{(table\_3way))}
\CommentTok{\#Breslow{-}Day test (tests for homogeneity of odds ratios across strata)}
\FunctionTok{BreslowDayTest}\NormalTok{(table\_3way\_safe)}
\end{Highlighting}
\end{Shaded}

\begin{verbatim}
## 
##  Breslow-Day test on Homogeneity of Odds Ratios
## 
## data:  table_3way_safe
## X-squared = 6.171, df = 2, p-value = 0.04571
\end{verbatim}

\textbf{So the effect of diabetes on resistance in fact appears and does
depend on bacteria type.} But as we remember from our heat map we
observed that there should not be any dependence. To be more rigorous,
let us check odds ratios of stratas and see if they diverge from 1 and
are different among themselves.

\begin{Shaded}
\begin{Highlighting}[]
\CommentTok{\#calculating odds ratios}
\NormalTok{or\_check }\OtherTok{\textless{}{-}}\NormalTok{ dataset\_readed }\SpecialCharTok{\%\textgreater{}\%}
  \FunctionTok{group\_by}\NormalTok{(souche\_description) }\SpecialCharTok{\%\textgreater{}\%}
  \FunctionTok{summarise}\NormalTok{(}

    \AttributeTok{A =} \FunctionTok{sum}\NormalTok{(Diabetes }\SpecialCharTok{==} \StringTok{"True"} \SpecialCharTok{\&}\NormalTok{ Resistance\_Status }\SpecialCharTok{==} \StringTok{"Resistant"}\NormalTok{),}
    \AttributeTok{B =} \FunctionTok{sum}\NormalTok{(Diabetes }\SpecialCharTok{==} \StringTok{"True"} \SpecialCharTok{\&}\NormalTok{ Resistance\_Status }\SpecialCharTok{==} \StringTok{"Susceptible"}\NormalTok{),}
    \AttributeTok{C =} \FunctionTok{sum}\NormalTok{(Diabetes }\SpecialCharTok{==} \StringTok{"No"} \SpecialCharTok{\&}\NormalTok{ Resistance\_Status }\SpecialCharTok{==} \StringTok{"Resistant"}\NormalTok{),}
    \AttributeTok{D =} \FunctionTok{sum}\NormalTok{(Diabetes }\SpecialCharTok{==} \StringTok{"No"} \SpecialCharTok{\&}\NormalTok{ Resistance\_Status }\SpecialCharTok{==} \StringTok{"Susceptible"}\NormalTok{),}
    
    \CommentTok{\#adding +0.5 to avoid division by zero (Haldane correction)}
    \AttributeTok{Odds\_Ratio =}\NormalTok{ ((A }\SpecialCharTok{+} \FloatTok{0.5}\NormalTok{) }\SpecialCharTok{*}\NormalTok{ (D }\SpecialCharTok{+} \FloatTok{0.5}\NormalTok{)) }\SpecialCharTok{/}\NormalTok{ ((B }\SpecialCharTok{+} \FloatTok{0.5}\NormalTok{) }\SpecialCharTok{*}\NormalTok{ (C }\SpecialCharTok{+} \FloatTok{0.5}\NormalTok{)),}
    
    \CommentTok{\#check sample size to see if heatmap was misleading}
    \AttributeTok{Total\_Patients =} \FunctionTok{n}\NormalTok{()}
\NormalTok{  )}

\FunctionTok{print}\NormalTok{(or\_check)}
\end{Highlighting}
\end{Shaded}

\begin{verbatim}
## # A tibble: 9 x 7
##   souche_description          A     B     C     D Odds_Ratio Total_Patients
##   <chr>                   <int> <int> <int> <int>      <dbl>          <int>
## 1 acinetobacter baumannii     2    35    12   104      0.589            153
## 2 citrobacter spp.           17    74    45   271      1.40             407
## 3 enterobacteria spp.        16   143    84   578      0.787            821
## 4 escherichia coli          667   396  2423  1572      1.09            5058
## 5 klebsiella pneumoniae      37    72   159   314      1.02             582
## 6 morganella morganii         2    53    11   188      0.766            254
## 7 proteus mirabilis           7   119    59   441      0.466            626
## 8 pseudomonas aeruginosa      2    33    16   113      0.513            164
## 9 serratia marcescens         1    37     3   175      2.01             216
\end{verbatim}

We see that in fact there are odds ratios that are far from 1 and
differ. So the heat map, although being a nice visual, does not
represent a relation that is indeed present.

\subsection{Conclusions}\label{conclusions}

Our project was mainly aimed for discovering patterns in antibiotic
resistance, approving theoretical knowledge from P\&S course and biology
with practical applications of testing hypothesis. It is known that
diabetes has an impact on immune system but we wanted to prove whether
this impact influences antibiotic resistance. First of all, diabetes in
itself does not directly raise resistance of infection.Though, we wanted
to observe the indirect influence describing by such sequence of events

\textbf{Diabetes} -\textgreater{} Weaker immune system -\textgreater{}
More frequent illnesses -\textgreater{} Consistent consumption of
antibiotics -\textgreater{} Bacteria adaptation to antibiotics
-\textgreater{} \textbf{Produced resistance}

In order to test it we needed to have an equal distributions of
resistant and susceptible age groups so that if observed we can conclude
that dependence is not biased (as the elderly people have more frequent
diabetes occurence). Indeed, the distributions was equal.

Hence, we discovered, using test for independence, that diabetes does
not raise resistance on the whole data. But then, splitting the data
groups into top 3 infections, the impact occurred.

Revising the work done so far, we inferred insightful knowledge about
outcomes of our results: Simpsons' paradox. That is a phenomenon in
probability and statistics where a trend or relationship that appears in
several different groups of data disappears or even reverses when the
groups are combined. In our specific example, the paradox means that
while our overall test on all bacteria groups showed no dependence
between diabetes and bacterial resistance, the relationship emerged when
we analyzed the top 3 groups of bacteria separately.

To sum up, in this mini-research project we have worked with practical
dataset from clinical research. We have examined key metrics (age) and
risk factors (diabetes). We have successfully checked resistance
patterns. Precisely, there is no pattern regarding age of patients,
however, there is a great pattern in having diabetes and specific type
of infection that through all data asserts resistance to that infection.
In future, to continue our research, possible consequent areas for
improvement could be building a logistic regression on classifying
bacterial infection (resistant/susceptible), examining dependence of
resistance to specific geographical area (South, East, West, North).

\end{document}
